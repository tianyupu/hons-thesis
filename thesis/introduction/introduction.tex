\chapter{Introduction}

\section{Background}
Suppose that we have a (perhaps large) data set. It could consist of
demographic data taken from a census, details of transactions in a shopping
centre, or simply a collection of text documents. The term used to describe
the methods of analysis used to find unknown relationships and summarise
information in these data sets is \textit{data mining} \citep{Hand2001}.
Broadly speaking,
data mining can be broken down into five areas (or \textit{tasks}) depending
on the aims of the person analysing the data:

\begin{enumerate}
\item Exploratory data analysis (EDA), which consists of simply exploring some
given data set without any particular goal or ideas of what to look for.
Methods in EDA usually involve interaction and visualisation, but this can
become difficult if there are too many data points;
\item Descriptive modelling, as its name suggests, seeks to describe the data
or the processes that generate it. The major areas here are clustering (finding
partitions of the data) and density estimation (finding the probability
distribution of the data);
\item Predictive modelling aims to create models that will allow the value of
some desired quantity or metric to be predicted based on some known or observed
values in the data set. The two major sub-areas of predictive modelling are
\textit{classification} and \textit{regression}, distinguished only by the
nature of the quantity we want to predict. If we want to predict whether
something belongs in a category, such as the disease of a patient, then we
are dealing with classification; otherwise, if we wish to predict a numeric
quantity, such as the value of the stock market at some future date, then
it is a regression problem;
\item Pattern and rule discovery, which covers anomaly detection (finding data
points which do not fit an expected trend, like detecting credit card fraud)
and association rule learning (discovering relationships between various items
in transaction databases, such as the food items purchased frequently together
in a supermarket);
\item Content retrieval, where the user wishes to find patterns in a data set
that are similar to the one he or she specifies. Search engines are a familiar
application of this area of data mining.
\end{enumerate}

The technical methods of data mining are drawn substantially from a field of
computer science known as \textit{machine learning}, which is dedicated to the
study of systems that can learn structural descriptions from data examples.
In this thesis we will focus on the task of predictive modelling, also
known as \textit{supervised learning} in the field of machine learning.
Specifically,
our problem will be one of classification. We will introduce the necessary
terminology in the next section as we describe the problem, as well as
throughout this thesis as necessary.

\section{Problem Statement}
Consider a data set of patients admitted to the trauma ward of a hospital.
This data set looks like a very large table: there are many columns, each one
describing some aspect of a patient (for example, age); there are also many
rows, each corresponding to a particular patient that was admitted. We will
refer to the columns as \textit{features} or \textit{attributes}, so-named
because they describe some particular aspect of the rows, which are often
called \textit{instances}, \textit{samples}, or \textit{training examples}.
Concretely, as an example, our data set could contain
two features (age and gender) for five patients: as a table, this means that
we have two columns, one for age and another for gender, with five rows
containing the age and gender data for each patient. Attributes can be numeric
(such as age) and potentially take on an infinite number of different values,
or categorical (such as gender) and can only have one of a limited, pre-defined
list of values.

In a general supervised learning problem, we wish to predict a value for some
desired quantity given some values for the attributes. The relationship between
the features and the desired quantity can be \textit{learned} from using data
that contains known values for the quantity that we wish to predict: this is
what makes the problem \textit{supervised}.
Our problem is a binary \textit{classification} task in which we
predict one of two possible categories for a patient: a length of stay (LOS) of
less than or equal to 2 days, or a LOS of greater than 2 days. These outcomes
are also called \textit{classes} or \textit{class labels}.

To do this, we use our data set of trauma patients with \textit{known}
values for the LOS and apply various \textit{learning} or
\textit{classification algorithms} to deduce
a relationship between the features and the LOS. Once the learning algorithm
or \textit{classifier}
has ``learned'' this relationship, it is then able to predict the class of
any patient given a set of features describing him or her.
Note that the LOS is simply another attribute of our data set; however, because
it is the quantity that we are interested in predicting, it is also called the
\textit{target}, \textit{outcome}, or \textit{class}.

\section{Motivation}
The LOS is an important measure of resource utilisation in hospitals around
the world \citep{Ng2006},
and many studies have been carried out to investigate whether the
LOS of a patient in various medical domains -- after surgery, burns, or
intensive care treatment to name a few -- can be predicted accurately.
Regardless of the
specifics of the problem, accurate LOS prediction results in better planning
and management of limited resources such as staff, equipment and medical
supplies. This is particularly critical as hospitals often face the problem
of limited funding \citep{Walczak2003}.
As we will see in the next chapter, there has been limited
application of a variety of data mining techniques to predict the LOS in
various medical domains, including trauma. Our work will aim to bridge this
gap by providing insights on the application of a wide range of data mining
techniques to LOS prediction.

\section{Contributions}
The work presented in this thesis aims to address several gaps in the current
research into LOS prediction, by building upon the work of Dinh et al.
\citep{Dinh2013a}. We systematically and empirically evaluated a comprehensive
catalogue of 11 classification
algorithms and 5 feature selection methods, as well as the effect of
discretisation, for predicting the LOS of trauma patients. 3 of the
classifiers and 4 of the feature selection methods have not been applied in
LOS prediction to date. Additionally, we are also not aware of any previous
work using the discretisation method that we evaluate in this thesis.

We also proposed a new distance function, Ranked Distance, for use with one
of the classification algorithms that we evaluate, and showed that this
performed better than the standard distance function in certain situations.

Finally, we conducted our evaluations on two independent data sets from
different medical domains: one from trauma and the other a general hospital
data set. Not only is the evaluation of a second data set in one work
uncommon in the literature, but the inclusion of more than one data set
allowed us to draw additional insights about the classifiers and feature
selections we used which would not have been possible with only one set of
data.

\section{Thesis Outline}
This thesis is structured as follows. Chapter \ref{chap:litreview} introduces
the major methods in supervised learning as used in past work on LOS
prediction. We explore the major algorithms, summarise the current state of
research, and identify where our work fits within this area.

Chapters \ref{chap:classification}, \ref{chap:preprocess} and
\ref{chap:selection} describe in detail the classification algorithms, and
feature preprocessing and selection methods used in our work. This leads into
the process of our evaluation, which is detailed in Chapter
\ref{chap:evaluation}.

Our findings are presented and discussed in Chapter \ref{chap:results}.
We conclude in Chapter \ref{chap:conclusion}, with indications for future
work in this area.
