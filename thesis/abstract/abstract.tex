\chapter*{Abstract}

Predicting the length of hospital stay for injured patients involves making a
judgement about how long they need to be hospitalised, so that doctors and
hospital staff can plan the appropriate course of treatment for the patient
to ensure they recover in the quickest possible time. This also ensures the
most efficient use of limited hospital resources. However, previous studies
have used a limited number of statistical techniques such as logistic
regression in order to predict the length of stay. Also, they have used only
manual feature selection and have not investigated the effect of automatic
feature selection on classifier performance.

Our work systematically evaluates and compares the performance of a number of
state-of-the-art classification algorithms and feature selection methods,
and also assesses the effect of feature discretisation for all combinations
of classifiers and feature selectors. Our evaluation were conducted on two
datasets, one from a trauma ward in a major Sydney hospital and another one
from a general hospital in Portugal. We propose a new classification algorithm
called Ranked Distance Nearest Neighbor that considers the
relative importance of each feature when determining the nearest neighbours.

For the trauma dataset, we were able to achieve an accuracy of 77.81\% with
1-nearest neighbour and an area under the curve of 0.846 with logistic
regression. This represents an improvement of 2.75\% and 0.034 respectively
over the previous best result on the same dataset. For the
general hospital dataset, the best results were 98.23\% accuracy with a
decision tree and 0.994 area under the curve with logistic regression,
support vector machine and K*. We also show that our proposed Ranked
Distance Nearest Neighbour improves accuracy and area under the curve over
traditional nearest neighbour algorithms. We comprehensively discuss
the results and suggest new directions for future studies.
