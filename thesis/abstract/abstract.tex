\chapter*{Abstract}

Predicting the length of hospital stay for injured patients involves making a
judgement about how long we believe they will need to be hospitalised, so that
doctors and hospital staff can plan the appropriate course of treatment for the
patient to ensure they recover in the quickest possible time. This ensures the
most efficient use of limited hospital resources. However, previous studies
have used a limited number of statistical techniques such as logistic
regression in order to predict the length of stay.

Our work systematically
evaluates the effect of discretisation on all combinations of 11 classifiers
using 5 different feature selection methods. One of these classifiers is a
novel contribution of our work. Furthermore, our evaluation includes 4 feature
selection methods and 3 classifiers which have to date not been used in
predicting the length of stay, compared alongside commonly used
techniques. We also compare two independent data sets.

Our results show that while logistic regression tended to perform best out of
many combinations, it is possible to achieve to achieve an accuracy and an area
under the receiver operating characteristic curve that is less than 2\% worse
than this best result using only around 15\% of the total features. We were
also able to improve upon the baseline accuracy and area under the curve by
2.75\% and 0.034 respectively, and recommend further investigation into the
use of nearest neighbour approaches and feature selection to future studies
of length of stay prediction.
