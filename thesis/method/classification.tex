\chapter{LOS Classification and Supervised Learning}
  \label{chap:classification}

In order to use a supervised learning algorithm to identify objects with
a particular classification, we first need \textit{feature vectors} to
describe these objects. In our case, each patient is converted into one
feature vector, where each \textit{feature} in this vector tells us
something about the
patient that can help us decide which LOS category he or she should be
placed in. The features comprising each feature vector must be the same,
but with different values. By comparing feature values between different
vectors, we are able to decide which \textit{class} or category a patient
belongs in. For example, if we wanted to classify the weather on a given
day as ``good'' or ``bad'', we could use two features -- temperature and
the amount of rain -- which are likely to be key indicators. Classifying
several days simply involves creating a feature vector that corresponds
to each day, with differing values for the temperature and the amount of
rain, and using these values to decide whether or not a day had good or
bad weather.

Since we will be using supervised learning techniques, we need
\textit{training examples}, which consist of a set of feature vectors
that are already labelled with the class that it belongs in. These
training examples are then used by a classifier to ``learn'' the
relationship linking the feature vectors to their class labels. In this
chapter we describe each of the learning algorithms we used to model the
relationship between the feature vectors and the LOS.

\section{ZeroR and OneR}

\section{Na\"{i}ve Bayes}

\section{Decision Tree (C4.5)}

\section{Logistic Regression}

\section{Support Vector Machine (SVM)}

\section{$k$-Nearest Neighbour}
\subsection{Basic Algorithm}

\subsection{Extensions}
\subsubsection{K*: an Entropy-Based Distance Function}

\subsubsection{Ranked Distance}

\section{Feed-forward Neural Network}

